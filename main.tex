\documentclass[12pt]{article}
\linespread{1.2}
\setlength{\parskip}{1em}
\usepackage{hyperref}
\usepackage{xpatch}
\usepackage[sorting=none]{biblatex}
\usepackage{fancyhdr}
\usepackage[a4paper, margin=1.25in]{geometry}
\addbibresource{references.bib}
\hypersetup{
    colorlinks=true,
    linkcolor=black,
    filecolor=magenta,      
    urlcolor=black,
    citecolor=black
    }
\urlstyle{same}

\title{}
\date{June 2025}

% Here you can find a backup of this project: 
% https://github.com/ThisisShoo/APS-1034-Team-Project-Report

% Lucid Chart: 
% https://lucid.app/lucidchart/f0b534e7-64c4-4566-ac8d-bf4807728f79/edit?invitationId=inv_bbe79a52-9dfe-407f-94ed-31a270cd020d&page=0_0#

%Links for timeline: 
%\href{https://www.cbc.ca/news/canada/sudbury/elliot-lake-mall-chronology-from-birth-to-death-and-beyond-1.2799462}{https://www.cbc.ca/news/canada/sudbury/elliot-lake-mall-chronology-from-birth-to-death-and-beyond-1.2799462}

%\href{https://globalnews.ca/news/402886/timeline-the-elliot-lake-mall-collapse/}{https://globalnews.ca/news/402886/timeline-the-elliot-lake-mall-collapse/}

\href{https://www.youtube.com/watch?v=2tO1L3SKhfU}{%https://www.youtube.com/watch?v=2tO1L3SKhfU}

\begin{document}
\maketitle


\newpage

\section*{Executive Summary}
% Purpose: Summarize the objectives, key findings, and recommendations of the analysis.

% Overview: Provide a brief description of the Algo Centre Mall collapse, including the date, location, and immediate consequences.

\section{Introduction} 
% - History of the mall, chain of events, as well as the methodology for analysis 
% Background: Detail the history of the Algo Centre Mall, including its construction in 1979-1980, architectural features like the rooftop parking, and its significance to the Elliot Lake community.
% Incident Summary: Outline the events leading up to the collapse on June 23, 2012, the collapse itself, and the immediate aftermath.
% Scope and Methodology: Explain the CAST methodology and its relevance to analyzing this particular incident.

The Algo Centre Mall, located in Elliot Lake, Ontario, was constructed between 1979 and 1980 by Algocen Realty Holdings Ltd. Housing a public library, a government service center, a hotel and retirement residence, and various retail stores, the mall served not only as a commercial hub but also as a community gathering space. The three-story structure was composed of concrete and steel-reinforced concrete slabs supported by a steel frame, sitting on a sloped terrain facing northeast. A parking deck was built on the roof, which was accessible via ramps along the hillside. The parking deck was supported by a steel frame held together by bolted and welded connections, with a layer of hollow core concrete slabs on top.

Throughout the years that immediately followed its construction, the mall was plagued by significant structural issues, particularly with water leaks from the rooftop parking deck, owing to failed waterproofing measures. Such water leaks, along with cracks that developed in the concrete slabs, were reported as early as 1981, and the building's condition continued to deteriorate over the years. Ten years later, a report prepared by Trow Consulting Engineers Ltd. found that although the rooftop parking deck was generally in good condition, various components of the building had visible signs of deterioration, including the aftermath of failed repairs, sections of broken concrete, and surface rust on exposed steel. In the Report of the Elliot Lake Commission of Inquiry (henceforth referred to as the Elliot Lake Report, or simply the Report) \cite{AlgoLakeReport1}, it was noted that the proposed waterproofing for the rooftop parking deck only met the requirements of the Ontario Building Code at that time on a technicality, for the 1975 code never specified what the waterproof membrane should be made of. Although a waterproof sealant was relied upon to act as a waterproofing membrane, it clearly failed to do so, as water leaks continued to be reported throughout the years. 

The aforementioned 1991 Trow report also indicated an abnormal amount of chloride content in the slabs, presumably from the de-icing salts used on the parking deck or brought in by vehicles, which proved to have exacerbated the corrosion of the steel components. Trow recommended the installation of a new waterproof membrane and a layer of asphalt to replace the existing concrete topping, but this recommendation was rejected by Algocen due to the high cost of repairs and the disruption of income from the mall's hotel. Testimonies from the Report indicated that Algocen Realty Holdings was financially capable of performing the renovations, but they never changed the way they dealt with the leakage, and eventually sold the mall to Elliot Lake Retirement Living as-is. During this transaction, Algocen did not provide Retirement Living with any of the engineering reports describing the structural issues and leaks.

The Elliot Lake Retirement Living (ELRL, or ``Retirement Living'') is a non-profit organization established in 1991 to promote Elliot Lake as a retirement community. Formerly a uranium mining town, Elliot Lake's population reached its peak in the 1980s due to the mining boom, but the closure of the mines in the early 1990s dealt a huge blow to the local economy, and the city's population has been in steady decline ever since \cite{ElliotLakePopulation}. With Retirement Living taking the center stage in the city's transformation into a retirement community, the organization bought many properties in Elliot Lake, including the Algo Centre Mall in 1999 from Algocen Realty Holdings, to convert them into retirement getaways \cite{NYT1996}. Retirement Living incorporated NorDev, a for-profit subsidiary, to manage the properties, including the mall and the hotel. Despite the close ties between Retirement Living and the city council, the organization was not obligated to share with the city any information that may harm the organization, including the reports on the state of the Algo Centre Mall. 

The Report also indicated that Retirement Living had no plans to fully address the roof deck issue. The mall's property manager, Mr. Richard Quinn, testified in front of the Commission counsel to further confirm that the organization simply continued the same practices as before it acquired the property, except the maintenance team simply became more adept at patching the leaks, but never at preventing those leaks from happening in the first place. Further, he testified that he interpreted the 1999 Halsall report, which was acquired by Retirement Living as it purchased the mall, as an endorsement of the ongoing maintenance practices. The Halsall report suggested two options for the roof deck: either find and seal all the cracks, or install a waterproof membrane. However, Halsall presented the second option with the requirement that all the cracks must be thoroughly found and sealed, though this point was not sufficiently emphasized. Having mistaken Halsall's suggestion as an endorsement for continuous maintenance, the mall's deterioration proceeded along its course through Retirement Living's ownership of the Algo Centre Mall. 

% Now talk about the Eastwood years. Basically, Bob Zazarian is a fraud


\section{Safety Control Structure}
% Control Hierarchy: Map out the safety control structure, detailing how safety constraints were supposed to be enforced from design through operation.

% Communication Channels: Examine the communication pathways among stakeholders and how information about structural issues was shared or neglected.

% Regulatory Oversight: Assess the role of municipal and provincial regulations in ensuring building safety and how these may have failed.

\underline{System Goals}

G1. Provide a safe, functional, and accessible commercial environment for the residents and visitors of Elliot Lake

G2. Ensure the structural integrity of the building throughout its lifecycle

\underline{Losses}

L1. Human losses: loss of life, physical injuries and psychological harm

L2. Material and structural losses: destruction of property and economic losses

L3. Institutional and professional failures: loss of public trust, damage to the image of the engineering profession, and less trust in governmental accountability

L4. Disruption to society: loss of a commercial space, and breakdown of emergency responses

\underline{Hazards}

H1. Corrosion of the structure due to leaking from the rooftop parking deck [L1, L2, L3, L4]

H2. Ineffective inspection and maintenance routine [L1, L2, L3, L4]

H3. Inadequate engineering assessments [L1, L2, L3, L4]

H4. Regulatory and oversight gaps [L1, L2, L3, L4]

H5. Unclear responsibility allocation and poor communication [L1, L2, L3, L4]

H6. Economic and political pressures [L4]

H7, Unpreparedness when an emergency emerges [L1, L3, L4]

\underline{System Safety Requirements and Constraints}

SR1: The building must be designed and maintained to ensure long-term structural integrity under expected environmental and usage conditions. [H1]

SC1: Structural components must not be allowed to degrade to the point of losing load-bearing capacity. [H1]

SR2: Regular, thorough inspections of critical infrastructure (e.g., roof, support beams) must be conducted by qualified professionals. [H2]

SC2: Identified structural deficiencies must be documented transparently and acted upon within a reasonable time frame. [H2]

SR3: Engineering assessments must be objective, evidence-based, and prioritize public safety over client interests. [H3]

SC3: Engineers must communicate risk levels clearly and explicitly, including in cases where conditions are unsafe or immediate action is needed. [H3]

SR4: Municipal and provincial authorities must have mechanisms to enforce that the building safety standards are followed. [H4]

SC4: Regulatory bodies must step in when reports indicate serious structural risks, including through orders, fines, or building closure. [H4]

SR5: Property owners must allocate resources for ongoing maintenance and be held accountable for not doing repairs that pose safety risks. [H5]

SC5: Cost-saving measures must not go above the minimum safety standards required for public occupancy. [H5]

SR6: Emergency response plans must account for building collapse scenarios and provide for safe, timely rescue efforts. [H6]

SC6: Emergency response should not be suspended without an effective alternative strategy when lives may be at risk. [H6]

SR7: The public and tenants must be informed about any significant structural safety risks affecting the buildings they use. [H7]

SC7: Safety-related information must not be kept away from the public due to commercial, political, or reputational concerns. [H7]




\section{Event Analysis (tentative name)} % include causal loop and misc diagrams. RAUL
%accimap or (walkerton model -> TERESA)
% Timeline of Events: Construct a detailed timeline from the mall's construction to the collapse, highlighting key events such as inspections, reports of leaks, and maintenance actions.


\begin{itemize}
    \item \textbf{\textbf{1976 }}
Algoma Central Properties asks architect James Keywan to design a new mall. The ultimate goal was to create a visually pleasing modern mall. 

    \item \textbf{\textbf{1979}}
The Ontario Municipal Board approves the construction of the Algo Centre Mall by Algocen Realty Holdings Ltd., the real estate branch of the Algoma Central Railway. The cost is estimated to be between \$10 million -\$12 million with John Kadlec as the structural engineer.

    \item \textbf{\textbf{1980}}
    \begin{itemize}
        \item August 5, 1980 =The architect and engineer sign the certificate of substantial completion for the Mall.
        \item The first tenants to occupy the mall include Woolco, Dominion, and Shoppers Drug Mart.
        \item Leaks start soon after opening 
    \end{itemize}
    \item \textbf{\textbf{1996 }}
A city report titled “Downtown, Core, and Industrial Area Improvements” is very critical of the overall design and aesthetic appearance of the Algo Centre Mall.

    \item \textbf{\textbf{1999 }}
The mall was sold to Elliot Lake Retirement Living.

    \item \textbf{\textbf{2005 }}
Bob Nazarian, under the name Eastwood Mall Inc., buys the mall for \$6.2 million.

    \item \textbf{\textbf{2006}}
October 24, 2006 = The City of Elliot Lake issues a Notice of Violation and Order to Conform to the Ontario Fire Code to the owners of the Mall.

    \item \textbf{\textbf{2007}}
Starlight Cafe sues Eastwood Incorporated, alleging in court documents that leaks above her shop twice caused the roof to collapse. Ultimately awarded \$11,000 in damages.

    \item \textbf{\textbf{2008}}
Bob Nazarian and Eastwood Incorporated announce plans to repair and renovate the roof of the Algo Centre Mall. John Clinckett (Architect)  is retained to install a protective membrane on the roof to keep water out (which had been part of the original 1979 design for the mall) as well as add a layer of asphalt to maintain rooftop parking.

    \item \textbf{\textbf{2008/2009}}
Nazarian did not accept  the \$1 million cost of repairs and cancelled the contract. Nazarian hired contractors Peak Building Restoration to complete repairs. They were noit paid the full amount owed to them by Nazarian. 

    \item \textbf{\textbf{2009}}
    \begin{itemize}
        \item September 25, 2009: The City of Elliot Lake issues an Order to Remedy under the City’s Property Standards By-law requiring that a series of deficiencies at the Mall be remedied.
        \item Fireproofing applied to a steel beam with a weld that ultimately failed
        \item Rust not reported
    \end{itemize}
    \item \textbf{\textbf{2010}}
    \begin{itemize}
        \item February 11, 2010: The City rescinds the property standards order.
        \item Engineering consulting firm Read Jones Christofferson Ltd. was hired to perform a study on retrofitting the mall. 
    \end{itemize}
    \item \textbf{\textbf{2011}}
A section of concrete crashes through the roof of Hungry Jack’s. Both mall management and the city are notified, but according to a restaurant employee the city’s inspector never arrived, and nothing is heard from Algo Centre management. (The restaurant is adjacent to the location of the 2012 roof collapse).

    \item \textbf{\textbf{2012}}
    \begin{itemize}
        \item March = Mall management pleaded guilty in court to having sprinklers and fire alarms that failed to code.
        \item April = Engineer Robert Wood makes an inspection and reports no structural damages and reports that “no visual signs of structural distress were observed” at the Mall.
        \item June 23rd = The roof of the Algo Centre mall collapses. Two cars fall through the 40-by-80-foot gap. The mall is evacuated, and two people are reported missing.
    \end{itemize}
\end{itemize}

% Failure Points: Identify specific points where safety constraints were breached, such as ignored engineering reports or inadequate repairs. RAUL

% Systemic Factors: Analyze how systemic issues, like organizational culture or economic pressures, contributed to the failure. RAUL

\section{Causal Factors and Contributing Conditions}
% Immediate Causes: Detail the direct technical reasons for the collapse, such as corrosion of structural components. KEVIN

% Underlying Systemic Causes: Explore deeper systemic issues, including decision-making processes, accountability structures, and resource allocations that allowed the immediate causes to develop. KEVIN

% Human and Organizational Factors: Assess how human behavior, organizational culture, and management decisions contributed to the incident. RAUL

\section{Recommendations}
% Design and Maintenance: Suggest improvements in building design practices and maintenance protocols to prevent similar failures. TERESA

% Regulatory Reforms: Propose changes to regulatory frameworks to enhance oversight and enforcement of safety standards. KEVIN

% Organizational Changes: Recommend organizational reforms to improve communication, accountability, and safety culture among stakeholders. RAUL

\section{Aftermath}
%TERESA

\section{Conclusion}
% Summary of Findings: Recap the key insights gained from the CAST analysis.

% Implications for Future Practice: Discuss how these findings can inform future practices in building design, maintenance, and regulation.

\section*{Appendices}

\newpage
\printbibliography
\end{document}
