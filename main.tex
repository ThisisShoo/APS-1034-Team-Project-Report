\documentclass[12pt]{article}
\linespread{1.2}
\setlength{\parskip}{1em}
\usepackage{hyperref}
\usepackage{xpatch}
\usepackage[sorting=none]{biblatex}
\usepackage{fancyhdr}
\usepackage[a4paper, margin=1.25in]{geometry}
\addbibresource{references.bib}
\hypersetup{
    colorlinks=true,
    linkcolor=black,
    filecolor=magenta,      
    urlcolor=black,
    citecolor=black
    }
\urlstyle{same}

\title{}
\date{June 2025}

% Here you can find a backup of this project: 
% https://github.com/ThisisShoo/APS-1034-Team-Project-Report

% Lucid Chart: 
% https://lucid.app/lucidchart/f0b534e7-64c4-4566-ac8d-bf4807728f79/edit?invitationId=inv_bbe79a52-9dfe-407f-94ed-31a270cd020d&page=0_0#

%Links for the timeline: 
%\href{https://www.cbc.ca/news/canada/sudbury/elliot-lake-mall-chronology-from-birth-to-death-and-beyond-1.2799462}{https://www.cbc.ca/news/canada/sudbury/elliot-lake-mall-chronology-from-birth-to-death-and-beyond-1.2799462}

% https://globalnews.ca/news/402886/timeline-the-elliot-lake-mall-collapse/

% https://www.youtube.com/watch?v=2tO1L3SKhfU

\begin{document}
\maketitle


\newpage

\section*{Executive Summary}
% Purpose: Summarize the objectives, key findings, and recommendations of the analysis.

% Overview: Provide a brief description of the Algo Centre Mall collapse, including the date, location, and immediate consequences.

\section{Introduction} 
% - History of the mall, chain of events, as well as the methodology for analysis 
% Background: Detail the history of the Algo Centre Mall, including its construction in 1979-1980, architectural features like the rooftop parking, and its significance to the Elliot Lake community.
% Incident Summary: Outline the events leading up to the collapse on June 23, 2012, the collapse itself, and the immediate aftermath.
% Scope and Methodology: Explain the CAST methodology and its relevance to analyzing this particular incident.

The Algo Centre Mall, located in Elliot Lake, Ontario, was constructed between 1979 and 1980 by Algocen Realty Holdings Ltd. Housing a public library, a government service center, a hotel and retirement residence, and various retail stores, the mall served not only as a commercial hub but also as a community gathering space. The three-story structure was composed of concrete and steel-reinforced concrete slabs supported by a steel frame, sitting on a sloped terrain facing northeast. A parking deck was built on the roof, which was accessible via ramps along the hillside. The parking deck was supported by a steel frame held together by bolted and welded connections, with a layer of hollow core concrete slabs on top. In 2012, a section of the roof deck collapsed after a long history of water leaks, structural deterioration, and, most importantly, neglect, resulting in the deaths of two people and over twenty injuries. This report aims to analyze the Algo Centre Mall collapse using the CAST (Causal Analysis based on System Theory) methodology, which focuses on understanding the systemic factors that contributed to the incident. The following section first provides a relatively brief history of the Algo Centre Mall and its deteriorating conditions leading up to the collapse, then gives an overview of the CAST methodology, and finally outlines the structure of this  report.

\subsection{History of the Algo Centre Mall}

Throughout the years that immediately followed its construction, the mall was plagued by significant structural issues, particularly with water leaks from the rooftop parking deck, owing to failed waterproofing measures. Such water leaks, along with cracks that developed in the concrete slabs, were reported as early as 1981, and the building's condition continued to deteriorate over the years. Ten years later, a report prepared by Trow Consulting Engineers Ltd. found that although the rooftop parking deck was generally in good condition, various components of the building had visible signs of deterioration, including the aftermath of failed repairs, sections of broken concrete, and surface rust on exposed steel. In the Report of the Elliot Lake Commission of Inquiry (henceforth referred to as the Elliot Lake Report) \cite{AlgoLakeReport1}, it was noted that the proposed waterproofing for the rooftop parking deck only met the requirements of the Ontario Building Code at that time on a technicality, for the 1975 code never specified what the waterproof membrane should be made of. Although a waterproof sealant was relied upon to act as a waterproofing membrane, it clearly failed to do so, as water leaks continued to be reported throughout the years. 

The aforementioned 1991 Trow report also indicated an abnormal amount of chloride content in the slabs, presumably from the de-icing salts used on the parking deck or brought in by vehicles, which proved to have exacerbated the corrosion of the steel components. Trow recommended the installation of a new waterproof membrane and a layer of asphalt to replace the existing concrete topping, but this recommendation was rejected by Algocen due to the high cost of repairs and the disruption of income from the mall's hotel. Testimonies from the Elliot Lake Report indicated that Algocen Realty Holdings was financially capable of performing the renovations, but they never changed the way they dealt with the leakage and eventually sold the mall to Elliot Lake Retirement Living as-is. During this transaction, Algocen did not provide Retirement Living with any of the engineering reports describing the structural issues and leaks. Algocen constructed the mall as they foresaw Elliot Lake's expansion in the 1980s, but quickly sought to offload its responsibilities to the mall as soon as the mining boom died. 

The Elliot Lake Retirement Living (ELRL, or ``Retirement Living'') is a non-profit organization established in 1991 to promote Elliot Lake as a retirement community. Formerly a uranium mining town, Elliot Lake's population reached its peak in the 1980s due to the mining boom, but the closure of the mines in the early 1990s dealt a huge blow to the local economy, and the city's population has been in steady decline ever since \cite{ElliotLakePopulation}. With Retirement Living taking the center stage in the city's transformation into a retirement community, the organization bought many properties in Elliot Lake, including the Algo Centre Mall in 1999 from Algocen Realty Holdings, to convert them into retirement getaways \cite{NYT1996}. Retirement Living incorporated NorDev, a for-profit subsidiary, to manage the properties, including the mall and the hotel. Despite the close ties between Retirement Living and the City Council, the organization was not obligated to share with the city any information that may harm the organization, including the reports on the state of the Algo Centre Mall. 

The Elliot Lake Report also indicated that Retirement Living had no plans to fully address the roof deck issue. The mall's property manager, Mr. Richard Quinn, testified in front of the Commission counsel to further confirm that the organization simply continued the same practices as before it acquired the property, except the maintenance team simply became more adept at patching the leaks, but never at preventing those leaks from happening in the first place. Further, he testified that he interpreted the 1999 Halsall report, which was acquired by Retirement Living as it purchased the mall, as an endorsement of the ongoing maintenance practices. The Halsall report suggested two options for the roof deck: either find and seal all the cracks, or install a waterproof membrane. However, Halsall presented the second option with the requirement that all the cracks must be thoroughly found and sealed, though this point was not sufficiently emphasized. Having mistaken Halsall's suggestion as an endorsement for continuous maintenance, the mall continued to utilize its own maintenance team instead of hiring qualified contractors, and its deterioration proceeded along its course through Retirement Living's ownership of the Algo Centre Mall. During this time, Retirement Living was in a good enough financial standing to perform a permanent fix to the roof deck, which was quoted by Halsall to cost \$776,000, but Retirement Living and NorDev instead spent \$1.3 million to attract Zellers, a major Canadian retailer, to the mall, as well as at least \$1.4 million in the golf course. Given the comparatively little amount that was really spent on the parking deck, as indicated in the Elliot Lake Report, it becomes clear that the 2012 collapse could have been avoided if Retirement Living and NorDev prioritized the safety of the building over the mall's business opportunities.

Eastwood Mall Inc. (henceforth referred to as ``Eastwood''), wholly owned by Mr. Bob Nazarian, would later purchase the mall in 2005 at a discount. It was later revealed in the Elliot Lake Report that Retirement Living was attempting to sell off the mall to Eastwood without disclosing the full extent of the mall's crumbling conditions, which was remarkably reminiscent of the deal between it and Algocen Realty Holdings. In his interview with the Commission, Mr. Nazarian commented as such about his other property purchases in August 2009: ``The Algo Mall was a black hole that no matter how much money you put in, [...] that mall was doomed.''\cite{AlgoLakeReport1}. Indeed, neither maintaining nor renovating the mall was going to be cheap long-term, and they would not have directly contributed to any financial returns, but the mall was still quite a profitable business, as indicated by the financial statements of Retirement Living, also shown in the Elliot Lake Report. This quote more goes to show how little Nazarian was interested in investing in the mall at that time. Instead of hiring outside engineering firms to conduct a thorough inspection of the mall, Nazarian relied on the Royal Bank of Canada's (RBC) engineers to assess it, as part of the loan application process before Nazarian could purchase the mall. The RBC engineers only conducted a visual inspection of the mall's structure and the equipment in the mechanical room. No structural components were inspected by a structural engineer, and no invasive tests were performed. As pointed out in the Elliot Lake Report, not conferring with outside consultation is not at all uncommon in Mr. Nazarian's business practices. This ignorance was further exacerbated by Retirement Living's refusal to disclose the structural issues from Eastwood, RBC, and the firm retained by RBC to survey the mall's condition, Construction Control. 

The fact that Eastwood's purchase turned out to be as problem-laden as the Algo Centre Mall may somewhat be attributed to sheer bad luck; one may also consider it amateurish or superficial in its efforts, as it failed to thoroughly inspect the mall before spending millions of dollars to purchase it. However, the sheer multitude of deliberate neglect and indifference that ensued after the purchase, which led to the eventual loss of two lives and injuries to many others, is simply inexcusable. In his testimony to the Commissioner, Mr. Nazarian indicated that he became aware of the building's structural damages since as early as October 2006, from yet another water leakage incident in the public library that led to a Notice of Violation issued by the City Council. On the other hand, Mr. Nazarian displayed a pattern of avoiding large expenditures from his company, especially on the matters pertaining to the roof repairs. Commissioner Bélanger noted in his report that Mr. Nazarian had a rich history of manipulating his firm's financial records to obtain loans, evade taxes, and to hide significant assets. With clearly self-serving motives, Mr. Nazarian repeatedly provided evasive or false testimonies to the Commission or to anyone involved in a business deal with him and Eastwood. One example is the creation of Empire Roofing, a shell company, to provide the facade of a legitimate roofing contractor. Through Empire Roofing, Eastwood was able to appear as if it had entered a contract to fix the roof, which would not only appease tenants such as Zellers but also allow Eastwood to make a dubious grant application. The said application was made possible by routing the payment through Empire Roofing to a legitimate contractor, Peak Restoration, who was hired to perform the work. Mr. Nazarian further reduced and delayed the payment to Peak Restoration, who had already started work with neither a contract signed by Eastwood nor a valid building permit. 

The above is not in any way an exhaustive list of Mr. Nazarian's questionable business practices in his evasion of accountability and his unethical pursuit of profit, but it is more than sufficient to demonstrate how little he invested in the already dire conditions of the Algo Centre Mall, both in terms of money and of corporate responsibility. Building on Commissioner Bélanger's conclusion to this segment of the Elliot Lake Report, Eastwood's failure to either sell off the mall or to prevent its collapse was a direct result of its own actions - its deceitful business maneuvers wore thin the trust of the tenants and lenders, its failure to deal honestly with its contractors ran the roof renovation project aground, and most importantly, it putting short-term profits above all else left the Algo Centre Mall to crumble.

% After Eastwood, use a paragraph to describe the contributions of the Ministry of Labor and the City Council, and then conclude this subsection

Although it is easy to blame the past owners for the 2012 collapse of Algo Centre Mall, a necessary component of this tragedy came from the regulatory bodies' failure to enforce the safety standards. The City Council of Elliot Lake had been officially made aware of the mall's ongoing water leakage issues since as early as 1991, when the public library made a complaint to the city. However, with the mall being a significant local hub of commerce and a community gathering space that had the same issue since its construction, it is inconceivable that the City Council had not been aware of the issues from the very beginning. Granted, the City of Elliot Lake may have been busy grappling with an existential crisis that occurred between the mall's construction and the turn of the century, but it had plenty of capacity to act after that. As a non-profit organization working closely with the City Council, Retirement Living was detrimental to revitalizing the city's economy as it transformed a former mining town into a successful retirement destination. However, this close relationship also meant a conflict of interest that ultimately impeded the City Council's ability to enforce safety standards. The City of Elliot Lake's Property Standards By-law includes a term that has remained valid since 1975, which mandates that the roof of a building and its drainage systems must be watertight, and that the maintenance of which is the responsibility of the owner. The Algo Centre Mall had already been leaking for decades, and no order for compliance was issued until after Retirement Living had already absolved itself of the responsibility to the mall. Even when it did issue orders against Eastwood, the City Council did not follow through with any enforcement actions, such as fines or closure of the mall, until the roof collapsed. % Bring up complaint-driven enforcements

The role of the Ministry of Labour (MoL) in the Algo Centre Mall collapse is also worth noting. The MoL had a field office in the mall in the early years of its operation, but the leaks still somehow went unnoticed or were taken unseriously until a few years before the collapse. % finish this paragraph

\subsection{CAST Methodology}


\section{Safety Control Structure}
% Control Hierarchy: Map out the safety control structure, detailing how safety constraints were supposed to be enforced from design through operation.

% Communication Channels: Examine the communication pathways among stakeholders and how information about structural issues was shared or neglected.

% Regulatory Oversight: Assess the role of municipal and provincial regulations in ensuring building safety and how these may have failed.

\underline{System Goals}

G1. Provide a safe, functional, and accessible commercial environment for the residents and visitors of Elliot Lake

G2. Ensure the structural integrity of the building throughout its lifecycle

\underline{Losses}

L1. Human losses: loss of life, physical injuries and psychological harm

L2. Material and structural losses: destruction of property and economic losses

L3. Institutional and professional failures: loss of public trust, damage to the image of the engineering profession, and less trust in governmental accountability

L4. Disruption to society: loss of a commercial space, and breakdown of emergency responses

\underline{Hazards}

H1. Corrosion of the structure due to leaking from the rooftop parking deck [L1, L2, L3, L4]

H2. Ineffective inspection and maintenance routine [L1, L2, L3, L4]

H3. Inadequate engineering assessments [L1, L2, L3, L4]

H4. Regulatory and oversight gaps [L1, L2, L3, L4]

H5. Unclear responsibility allocation and poor communication [L1, L2, L3, L4]

H6. Economic and political pressures [L4]

H7, Unpreparedness when an emergency emerges [L1, L3, L4]

\underline{System Safety Requirements and Constraints}

SR1: The building must be designed and maintained to ensure long-term structural integrity under expected environmental and usage conditions. [H1]

SC1: Structural components must not be allowed to degrade to the point of losing load-bearing capacity. [H1]

SR2: Regular, thorough inspections of critical infrastructure (e.g., roof, support beams) must be conducted by qualified professionals. [H2]

SC2: Identified structural deficiencies must be documented transparently and acted upon within a reasonable time frame. [H2]

SR3: Engineering assessments must be objective, evidence-based, and prioritize public safety over client interests. [H3]

SC3: Engineers must communicate risk levels clearly and explicitly, including in cases where conditions are unsafe or immediate action is needed. [H3]

SR4: Municipal and provincial authorities must have mechanisms to enforce that the building safety standards are followed. [H4]

SC4: Regulatory bodies must step in when reports indicate serious structural risks, including through orders, fines, or building closure. [H4]

SR5: Property owners must allocate resources for ongoing maintenance and be held accountable for not doing repairs that pose safety risks. [H5]

SC5: Cost-saving measures must not go above the minimum safety standards required for public occupancy. [H5]

SR6: Emergency response plans must account for building collapse scenarios and provide for safe, timely rescue efforts. [H6]

SC6: Emergency response should not be suspended without an effective alternative strategy when lives may be at risk. [H6]

SR7: The public and tenants must be informed about any significant structural safety risks affecting the buildings they use. [H7]

SC7: Safety-related information must not be kept away from the public due to commercial, political, or reputational concerns. [H7]




\section{Event Analysis (tentative name)} % include causal loop and misc diagrams. RAUL
%accimap or (walkerton model -> TERESA)
% Timeline of Events: Construct a detailed timeline from the mall's construction to the collapse, highlighting key events such as inspections, reports of leaks, and maintenance actions.

TIMELINE

\begin{itemize}
    \item \textbf{\textbf{1976 }}
Algoma Central Properties asks architect James Keywan to design a new mall. The ultimate goal was to create a visually pleasing modern mall. 

    \item \textbf{\textbf{1979}}
The Ontario Municipal Board approves the construction of the Algo Centre Mall by Algocen Realty Holdings Ltd., the real estate branch of the Algoma Central Railway. The cost is estimated to be between \$10 million -\$12 million with John Kadlec as the structural engineer.

    \item \textbf{\textbf{1980}}
    \begin{itemize}
        \item August 5, 1980 =The architect and engineer sign the certificate of substantial completion for the Mall.
        \item The first tenants to occupy the mall include Woolco, Dominion, and Shoppers Drug Mart.
        \item Leaks start soon after opening 
    \end{itemize}
    \item \textbf{\textbf{1996 }}
A city report titled “Downtown, Core, and Industrial Area Improvements” is very critical of the overall design and aesthetic appearance of the Algo Centre Mall.

    \item \textbf{\textbf{1999 }}
The mall was sold to Elliot Lake Retirement Living.

    \item \textbf{\textbf{2005 }}
Bob Nazarian, under the name Eastwood Mall Inc., buys the mall for \$6.2 million.

    \item \textbf{\textbf{2006}}
October 24, 2006 = The City of Elliot Lake issues a Notice of Violation and Order to Conform to the Ontario Fire Code to the owners of the Mall.

    \item \textbf{\textbf{2007}}
Starlight Cafe sues Eastwood Incorporated, alleging in court documents that leaks above her shop twice caused the roof to collapse. Ultimately awarded \$11,000 in damages.

    \item \textbf{\textbf{2008}}
Bob Nazarian and Eastwood Incorporated announce plans to repair and renovate the roof of the Algo Centre Mall. John Clinckett (Architect)  is retained to install a protective membrane on the roof to keep water out (which had been part of the original 1979 design for the mall) as well as add a layer of asphalt to maintain rooftop parking.

    \item \textbf{\textbf{2008/2009}}
Nazarian did not accept  the \$1 million cost of repairs and cancelled the contract. Nazarian hired contractors Peak Building Restoration to complete repairs. They were noit paid the full amount owed to them by Nazarian. 

    \item \textbf{\textbf{2009}}
    \begin{itemize}
        \item September 25, 2009: The City of Elliot Lake issues an Order to Remedy under the City’s Property Standards By-law requiring that a series of deficiencies at the Mall be remedied.
        \item Fireproofing applied to a steel beam with a weld that ultimately failed
        \item Rust not reported
    \end{itemize}
    \item \textbf{\textbf{2010}}
    \begin{itemize}
        \item February 11, 2010: The City rescinds the property standards order.
        \item Engineering consulting firm Read Jones Christofferson Ltd. was hired to perform a study on retrofitting the mall. 
    \end{itemize}
    \item \textbf{\textbf{2011}}
A section of concrete crashes through the roof of Hungry Jack’s. Both mall management and the city are notified, but according to a restaurant employee the city’s inspector never arrived, and nothing is heard from Algo Centre management. (The restaurant is adjacent to the location of the 2012 roof collapse).

    \item \textbf{\textbf{2012}}
    \begin{itemize}
        \item March = Mall management pleaded guilty in court to having sprinklers and fire alarms that failed to code.
        \item April = Engineer Robert Wood makes an inspection and reports no structural damages and reports that “no visual signs of structural distress were observed” at the Mall.
        \item June 23rd = The roof of the Algo Centre mall collapses. Two cars fall through the 40-by-80-foot gap. The mall is evacuated, and two people are reported missing.
    \end{itemize}
\end{itemize}

% Failure Points: Identify specific points where safety constraints were breached, such as ignored engineering reports or inadequate repairs. RAUL

FAILURE POINTS

Ignored Persistent Warnings = Leaks and other issues began shortly after the mall opened. This was just one of the early indicators of the roof's poor structure. Not only did they ignore the vast number of leaks, but they also ignored a great warning in June of 2011 when a piece of concrete fell from the roof at the restaurant Hungry Jack. Even though there was an initial report filed, an inspector never arrived to do the inspection, allowing the negligent management to continue like no wrong was done. 
 
Inadequate and Superficial roof repairs = Even with decades of complaints from various stakeholders, including customers and shops themselves, the mall management decided to use temporary and weak solutions to fix these issues instead of addressing the actual cause of the issues. As the saying goes, they were curing the symptoms but not the disease itself. As can be seen in the photo below, they were using nets to stop the leaks instead of addressing the root problem itself. This was due to the higher cost of more permanent and effective solutions.
  

 

 
Ignored / Rescinded city orders = Eliot Lake issued multiple warnings regarding the safety of customers and tenants. These warnings not only included the safety of these stakeholders but also included the maintenance of various issues. There was a notice of violation under the fire code and an order to remedy property standards. There were measures put in place due to these notices, but they were either inadequately enforced or built in order to be temporary solutions.  This bad application of city order allowed the structural issues to persist and gave a false sense of safety for both management and the public. 

 

 

 
Inadequate structural interventions = When the structural issues were being addressed, the repairs were either badly planned, insufficient, or incomplete. An example of this was how fireproofing was applied directly over the corroded beams without fixing those. In 2008/2009, there were plans to install a proper waterproofing membrane, but it was canceled due to cost concerns. It could be said that these fixes were mostly cosmetic rather than structural since they failed to address the growing corrosion problem. 

 

 
Superficial Engineering Report by Robert Wood = In April of 2012 (two months before the roof collapse), engineer Robert Wood signed a report stating that there were no visual structural issues visible at the mall. It was later found that he never removed any finishes or investigated areas that were prone to corrosion. He was pressured by the mall’s owner to downplay his concerns which directly contributed to the false sense of safety, delaying any action that could have prevented the collapse. 

 

 
Unchecked and widespread corrosion = Over 3o years, water leaks from the rooftop parking area caused steel beams and welds to rust severely. Investigations after the collapse indicated that some welds had corroded to only 12\% of their original strength and that approximately 40\% of structural connections showed deterioration. This corrosion remained unchecked, unreported, and ignored by the inspector who failed to identify the great danger this posed. 

 

 
Roof Design flaws = Due to the idea that the mall had to be modern and visually pleasing, the original mall design included a rooftop parking above the retail stores, which exposed the mall's internal structure to water and salt intrusion. A waterproofing membrane that was recommended by the architect was omitted entirely in the construction. Over the years, the roof was subject to unorganized vehicular loads, which combined with snow accumulation and salt exposure, increased the wear and corrosion rate.

 

 
Faulty Material installation = Waterproofing materials and concrete slabs were often installed under poor conditions, such as cold or wet weather, which compromises their durability and effectiveness. From the beginning, there were construction shortcuts taken that contributed to the degradation of the structure.  Another example of this is how they used hollow-core slabs in a high-exposure environment, such as the rooftop parking. Without proper and durable waterproofing, the integrity of these slabs was reduced significantly. The lack of reinforcement of these slabs made the collapse only a matter of time. 

 

 

 
Uncontrolled vehicle loads = The rooftop parking was a source of structural overload. Vehicle traffic on the roof created weight loads that the roof was not constructed to support. Reports are indicating that heavy and unregulated traffic (snowplows and delivery trucks) used the roof without structural reinforcements in place. 

 

 
Fabricated Maintenance records = Mall management and ownership (especially Bob Nazarian) submitted falsified maintenance invoices to the bank and lenders. These false documents included fake documentation for repairs that were never completed. This deception misled financial institutions but also prevented authorities from identifying and responding to risks from the mall's real conditions. 

% Systemic Factors: Analyze how systemic issues, like organizational culture or economic pressures, contributed to the failure. RAUL

SYSTEMIC FACTORS

· Cost-cutting over safety = Throughout the mall's history, the theme of prioritizing saving money/profits over ensuring safety was a recurring occurrence. The most notable example is how its owner, Bob Nazarian, cancelled a \$1 million contract to install the proper waterproofing system since it seemed too expensive. To save money, he hired a contractor who gave them a lower price and even failed to pay them in full after the repairs were completed (which may also speak to the quality of said repairs). These decisions accelerated deterioration and directly compromised the structural integrity of the parking lot. 

 

 · Weak regulatory oversight =The city of Elliot Lake failed to consistently enforce safety standards. Even after issuing multiple violations and hearing complaints from various sources, the city rescinded enforcement actions that they had placed previously, allowing Nazarian to neglect necessary repairs. During this time, inspectors failed to ensure code compliance. The leniency allowed unsafe conditions that remained unchecked and escalated through time. 

 

 · Organizational negligence and poor maintenance culture = There was a long history of problems that were either ignored, patched superficially, or never even attempted to be repaired. It is clear how the mall was under a management team that treated repairs as something reactive instead of proactive. An example of this is how leaks were addressed with caulking instead of fixing the root cause of the leak. The complaints from various tenants were dismissed, which created a culture of negligence when it came to repairs and maintenance. 

 

 · Weak communication among stakeholders = There was a systemic failure when communicating between vital stakeholders, including previous and current mall owners, engineers, city officials, and tenants. It is clear how the previous owners of the mall knew exactly the deficiencies of the structure and wanted to get rid of the liability. They assured the buyers the leaks were a small, fixable issue instead of the reality of the situation. Tenants reported major issues (like the piece of concrete that fell, and even though there was a complaint, the authorities never showed up,) and there was no follow-up from the mall. The broken communications between stakeholders led to no action being taken from authorities or other governing bodies who could have acted. 

 

 · Ethical lapses among professionals = Several professional engineers involved in the mall ignored their duty to public safety. Robert Wood admitted he did not conduct a proper inspection and not only downplayed his concerns but also included false reassurances in his report. This lapse can be seen in multiple levels of the organization, starting with Nazarian, who had no moral ground and decided that the safety of the community was secondary to his profits. The use of hollow-core slabs is proof that professionals who know the limitations of the materials used, especially in humid/cold environments, show how much the lapse of judgement occurred. Whether this was for cost-saving measures, poor maintenance culture, or an inspector, everyone who knew the limitations of the materials and that they were not optimal for the environment where being used had failed in their professional duty either on purpose or by accident. 

 

 · Surface-level inspections = Many of the inspections that were conducted of the structure and condition of the mall were done to a superficial level. The inspectors did not remove ceiling panels, and most importantly, they never tested for corrosion of the internal structure. All this allowed for deteriorated welds and extensive rust to go unnoticed and undocumented for decades. The mall was repeatedly declared structurally in good condition even with obvious external and internal issues to the structure.  

 

 

 


 · Profit-driven concealment = Nazarian and previous mall owners hid the truth about the mall's conditions to maintain financial viability. As previously mentioned, there is evidence of falsified invoices to lenders to make the appearance that repairs had been implemented. Negative findings during inspections were either omitted from reports or softened so that the mall could continue to operate in its current conditions without having to make repairs and maintenance. By hiding the true structural deficiency of the mall and its risks, management protected the short-term financial viability of the mall while exposing the community to increasing danger. 

 

 · Inadequate regulatory standards = The Ontario building code and inspection framework did not require invasive structural changes or account for specific risks (such as the rooftop parking). At the time, there was no regulatory guidance for long-term corrosion or structural integrity in environments with repeated water infiltration. Without the mandatory corrosion inspection or structural health monitoring, regulators had limited tools to enforce deeper reviews even with obvious and apparent structural issues. The 2012 Ontario Building Code was effective from January 1st, 2014. This code included provisions addressing corrosion, which is expected to be upheld in subsequent versions. 


\section{Causal Factors and Contributing Conditions}
% Immediate Causes: Detail the direct technical reasons for the collapse, such as corrosion of structural components. KEVIN

% Underlying Systemic Causes: Explore deeper systemic issues, including decision-making processes, accountability structures, and resource allocations that allowed the immediate causes to develop. KEVIN

% Human and Organizational Factors: Assess how human behavior, organizational culture, and management decisions contributed to the incident. RAUL

\section{Recommendations}
% Design and Maintenance: Suggest improvements in building design practices and maintenance protocols to prevent similar failures. TERESA

% Regulatory Reforms: Propose changes to regulatory frameworks to enhance oversight and enforcement of safety standards. KEVIN

% Organizational Changes: Recommend organizational reforms to improve communication, accountability, and safety culture among stakeholders. RAUL

\section{Aftermath}
%TERESA

\section{Conclusion}
% Summary of Findings: Recap the key insights gained from the CAST analysis.

% Implications for Future Practice: Discuss how these findings can inform future practices in building design, maintenance, and regulation.

\section*{Appendices}

\newpage
\printbibliography
\end{document}
