\documentclass[12pt]{article}
\linespread{1.2}
\setlength{\parskip}{1em}
\usepackage{hyperref}
\usepackage{xpatch}
\usepackage[sorting=none]{biblatex}
\usepackage{fancyhdr}
\usepackage[a4paper, margin=1.25in]{geometry}
\addbibresource{references.bib}
\hypersetup{
    colorlinks=true,
    linkcolor=black,
    filecolor=magenta,      
    urlcolor=black,
    citecolor=black
    }
\urlstyle{same}

\title{}
\date{June 2025}

% Here you can find a backup of this project: 
% https://github.com/ThisisShoo/APS-1034-Team-Project-Report

% Lucid Chart: 
% https://lucid.app/lucidchart/4e7d9b66-8e03-4d3c-bbfb-40a45487144a/edit?viewport_loc=-784%2C-1980%2C6596%2C3320%2C0_0&invitationId=inv_5fac3005-5c65-482d-b08d-a74f334d73be

% Drive Doc = https://docs.google.com/document/d/1SqCwM1th-N_Kp87VMr64HIsHCJK0QRoMJO3AX74vI7E/edit?usp=sharing

%Links for the timeline: 
%\href{https://www.cbc.ca/news/canada/sudbury/elliot-lake-mall-chronology-from-birth-to-death-and-beyond-1.2799462}{https://www.cbc.ca/news/canada/sudbury/elliot-lake-mall-chronology-from-birth-to-death-and-beyond-1.2799462}

% https://globalnews.ca/news/402886/timeline-the-elliot-lake-mall-collapse/

% https://www.youtube.com/watch?v=2tO1L3SKhfU

\begin{document}
\maketitle

\newpage

\tableofcontents
\newpage

\section*{Executive Summary}
% Purpose: Summarize the objectives, key findings, and recommendations of the analysis.

% Overview: Provide a brief description of the Algo Centre Mall collapse, including the date, location, and immediate consequences.

\section{Introduction} 
% - History of the mall, chain of events, as well as the methodology for analysis 
% Background: Detail the history of the Algo Centre Mall, including its construction in 1979-1980, architectural features like the rooftop parking, and its significance to the Elliot Lake community.
% Incident Summary: Outline the events leading up to the collapse on June 23, 2012, the collapse itself, and the immediate aftermath.
% Scope and Methodology: Explain the CAST methodology and its relevance to analyzing this particular incident.

The Algo Centre Mall, located in Elliot Lake, Ontario, was constructed between 1979 and 1980 by Algocen Realty Holdings Ltd. Housing a public library, a government service center, a hotel and retirement residence, and various retail stores, the mall served not only as a commercial hub but also as a community gathering space. The three-story structure was composed of concrete and steel-reinforced concrete slabs supported by a steel frame, sitting on a sloped terrain facing northeast. A parking deck was built on the roof, which was accessible via ramps along the hillside. The parking deck was supported by a steel frame held together by bolted and welded connections, with a layer of hollow core concrete slabs on top. In 2012, a section of the roof deck collapsed after a long history of water leaks, structural deterioration, and, most importantly, neglect, resulting in the deaths of two people and over twenty injuries. This report aims to analyze the Algo Centre Mall collapse using the CAST (Causal Analysis based on System Theory) methodology, which focuses on understanding the systemic factors that contributed to the incident. The following section first provides a relatively brief history of the Algo Centre Mall and its deteriorating conditions leading up to the collapse, then gives an overview of the CAST methodology, and finally outlines the structure of this  report.

\subsection{History of the Algo Centre Mall}

\subsubsection{Algocen built a defective mall amid a mining rush, sold its problems away at the sight of an economic downturn}

Throughout the years that immediately followed its construction, the mall was plagued by significant structural issues, particularly with water leaks from the rooftop parking deck, owing to failed waterproofing measures. Such water leaks, along with cracks that developed in the concrete slabs, were reported as early as 1981, and the building's condition continued to deteriorate over the years. Ten years later, a report prepared by Trow Consulting Engineers Ltd. found that although the rooftop parking deck was generally in good condition, various components of the building had visible signs of deterioration, including the aftermath of failed repairs, sections of broken concrete, and surface rust on exposed steel. In the Report of the Elliot Lake Commission of Inquiry (henceforth referred to as the Elliot Lake Report) \cite{AlgoLakeReport1}, it was noted that the proposed waterproofing for the rooftop parking deck only met the requirements of the Ontario Building Code at that time on a technicality, for the 1975 code never specified what the waterproof membrane should be made of. Although a waterproof sealant was relied upon to act as a waterproofing membrane, it clearly failed to do so, as water leaks continued to be reported throughout the years. 

The aforementioned 1991 Trow report also indicated an abnormal amount of chloride content in the slabs, presumably from the de-icing salts used on the parking deck or brought in by vehicles, which proved to have exacerbated the corrosion of the steel components. Trow recommended the installation of a new waterproof membrane and a layer of asphalt to replace the existing concrete topping, but this recommendation was rejected by Algocen due to the high cost of repairs and the disruption of income from the mall's hotel. Testimonies from the Elliot Lake Report indicated that Algocen Realty Holdings was financially capable of performing the renovations, but they never changed the way they dealt with the leakage and eventually sold the mall to Elliot Lake Retirement Living as-is. During this transaction, Algocen did not provide Retirement Living with any of the engineering reports describing the structural issues and leaks. Algocen constructed the mall as they foresaw Elliot Lake's expansion in the 1980s, but quickly sought to offload its responsibilities to the mall as soon as the mining boom died. 

\subsubsection{Retirement Living helped change the town, but not the mall maintenance practices}

The Elliot Lake Retirement Living (ELRL, or ``Retirement Living'') is a non-profit organization established in 1991 to promote Elliot Lake as a retirement community. Formerly a uranium mining town, Elliot Lake's population reached its peak in the 1980s due to the mining boom, but the closure of the mines in the early 1990s dealt a huge blow to the local economy, and the city's population has been in steady decline ever since \cite{ElliotLakePopulation}. With Retirement Living taking the center stage in the city's transformation into a retirement community, the organization bought many properties in Elliot Lake, including the Algo Centre Mall in 1999 from Algocen Realty Holdings, to convert them into retirement getaways \cite{NYT1996}. Retirement Living incorporated NorDev, a for-profit subsidiary, to manage the properties, including the mall and the hotel. Despite the close ties between Retirement Living and the City Council, the organization was not obligated to share with the city any information that may harm the organization, including the reports on the state of the Algo Centre Mall. 

The Elliot Lake Report also indicated that Retirement Living had no plans to fully address the roof deck issue. The mall's property manager, Mr. Richard Quinn, testified in front of the Commission counsel to further confirm that the organization simply continued the same practices as before it acquired the property, except the maintenance team simply became more adept at patching the leaks, but never at preventing those leaks from happening in the first place. Further, he testified that he interpreted the 1999 Halsall report, which was acquired by Retirement Living as it purchased the mall, as an endorsement of the ongoing maintenance practices. The Halsall report suggested two options for the roof deck: either find and seal all the cracks, or install a waterproof membrane. However, Halsall presented the second option with the requirement that all the cracks must be thoroughly found and sealed, though this point was not sufficiently emphasized. Having mistaken Halsall's suggestion as an endorsement for continuous maintenance, the mall continued to utilize its own maintenance team instead of hiring qualified contractors, and its deterioration proceeded along its course through Retirement Living's ownership of the Algo Centre Mall. During this time, Retirement Living was in a good enough financial standing to perform a permanent fix to the roof deck, which was quoted by Halsall to cost \$776,000, but Retirement Living and NorDev instead spent \$1.3 million to attract Zellers, a major Canadian retailer, to the mall, as well as at least \$1.4 million in the golf course. Given the comparatively little amount that was really spent on the parking deck, as indicated in the Elliot Lake Report, it becomes clear that the 2012 collapse could have been avoided if Retirement Living and NorDev prioritized the safety of the building over the mall's business opportunities.

\subsubsection{Duplicitous businessman purchased the mall without a thorough inspection, unwilling to fix but unable to sell}

Eastwood Mall Inc. (henceforth referred to as ``Eastwood''), wholly owned by Mr. Bob Nazarian, would later purchase the mall in 2005 at a discount. It was later revealed in the Elliot Lake Report that Retirement Living was attempting to sell off the mall to Eastwood without disclosing the full extent of the mall's crumbling conditions, which was remarkably reminiscent of the deal between it and Algocen Realty Holdings. In his interview with the Commission, Mr. Nazarian commented as such about his other property purchases in August 2009: ``The Algo Mall was a black hole that no matter how much money you put in, [...] that mall was doomed.''\cite{AlgoLakeReport1}. Indeed, neither maintaining nor renovating the mall was going to be cheap long-term, and they would not have directly contributed to any financial returns, but the mall was still quite a profitable business, as indicated by the financial statements of Retirement Living, also shown in the Elliot Lake Report. This quote more goes to show how little Nazarian was interested in investing in the mall at that time. Instead of hiring outside engineering firms to conduct a thorough inspection of the mall, Nazarian relied on the Royal Bank of Canada's (RBC) engineers to assess it, as part of the loan application process before Nazarian could purchase the mall. The RBC engineers only conducted a visual inspection of the mall's structure and the equipment in the mechanical room. No structural components were inspected by a structural engineer, and no invasive tests were performed. As pointed out in the Elliot Lake Report, not conferring with outside consultation is not at all uncommon in Mr. Nazarian's business practices. This ignorance was further exacerbated by Retirement Living's refusal to disclose the structural issues from Eastwood, RBC, and the firm retained by RBC to survey the mall's condition, Construction Control. 

The fact that Eastwood's purchase turned out to be as problem-laden as the Algo Centre Mall may somewhat be attributed to sheer bad luck; one may also consider it amateurish or superficial in its efforts, as it failed to thoroughly inspect the mall before spending millions of dollars to purchase it. However, the sheer multitude of deliberate neglect and indifference that ensued after the purchase, which led to the eventual loss of two lives and injuries to many others, is simply inexcusable. In his testimony to the Commissioner, Mr. Nazarian indicated that he became aware of the building's structural damages since as early as October 2006, from yet another water leakage incident in the public library that led to a Notice of Violation issued by the City Council. On the other hand, Mr. Nazarian displayed a pattern of avoiding large expenditures from his company, especially on the matters pertaining to the roof repairs. Commissioner Bélanger noted in his report that Mr. Nazarian had a rich history of manipulating his firm's financial records to obtain loans, evade taxes, and to hide significant assets. With clearly self-serving motives, Mr. Nazarian repeatedly provided evasive or false testimonies to the Commission or to anyone involved in a business deal with him and Eastwood. One example is the creation of Empire Roofing, a shell company, to provide the facade of a legitimate roofing contractor. Through Empire Roofing, Eastwood was able to appear as if it had entered a contract to fix the roof, which would not only appease tenants such as Zellers but also allow Eastwood to make a dubious grant application. The said application was made possible by routing the payment through Empire Roofing to a legitimate contractor, Peak Restoration, who was hired to perform the work. Mr. Nazarian further reduced and delayed the payment to Peak Restoration, who had already started work with neither a contract signed by Eastwood nor a valid building permit. 

The above is not in any way an exhaustive list of Mr. Nazarian's questionable business practices in his evasion of accountability and his unethical pursuit of profit, but it is more than sufficient to demonstrate how little he invested in the already dire conditions of the Algo Centre Mall, both in terms of money and of corporate responsibility. Building on Commissioner Bélanger's conclusion to this segment of the Elliot Lake Report, Eastwood's failure to either sell off the mall or to prevent its collapse was a direct result of its own actions - its deceitful business maneuvers wore thin the trust of the tenants and lenders, its failure to deal honestly with its contractors ran the roof renovation project aground, and most importantly, it putting short-term profits above all else left the Algo Centre Mall to crumble.

\subsubsection{Regulatory agencies are also to blame}

Although it is easy to blame the past owners for the 2012 collapse of Algo Centre Mall, a necessary component of this tragedy came from the regulatory bodies' failure to enforce the safety standards. The City Council of Elliot Lake had been officially made aware of the mall's ongoing water leakage issues since as early as 1991, when the public library made a complaint to the city. However, with the mall being a significant local hub of commerce and a community gathering space that had the same issue since its construction, it is inconceivable that the City Council had not been aware of the issues from the very beginning. Granted, the City of Elliot Lake may have been busy grappling with an existential crisis that occurred between the mall's construction and the turn of the century, but it had plenty of capacity to act after that. As a non-profit organization working closely with the City Council, Retirement Living was detrimental to revitalizing the city's economy as it transformed a former mining town into a successful retirement destination. However, this close relationship also meant a conflict of interest that ultimately impeded the City Council's ability to enforce safety standards. The City of Elliot Lake's Property Standards By-law includes a term that has remained valid since 1975, which mandates that the roof of a building and its drainage systems must be watertight, and that the maintenance of which is the responsibility of the owner. The Algo Centre Mall had already been leaking for decades, and no order for compliance was issued until after Retirement Living had already absolved itself of the responsibility to the mall. Even when it did issue orders against Eastwood, the City Council did not follow through with any enforcement actions, such as fines or closure of the mall, until the roof collapsed. The Elliot Lake Report pointed out that the enforcement were carried out in a complaint-driven manner, but given the time frame and what little was done about the mounting complaints, it clearly would have taken a lot more for the city to take any substantial action. 


\subsection{CAST Methodology}

Causal Analysis based on System Theory (CAST) is a methodology that focuses on understanding the systemic factors that contribute to incidents, rather than distributing blame to individual actors or components. Through analyzing the interactions between various components of a system, including the human, technical, and organizational elements, the CAST methodology is able to detach itself from the chronology of events and treat safety as an emergent property of the system. CAST is especially useful for complex indicents like the Algo Centre Mall collapse, where a faulty design was magnified by the cumulative effect of multiple factors and stakeholders over a period of three decades. 

The first step towards applying the CAST methodology is to build a comprehensive understanding of the system, including its goals, losses, hazards, safety parameters, and proximate events that led to the incident. This helps to construct a Safety Control Structure (SCS) that identifies the interaction pathways between the various components of the system, both at the time of the incident and as originally intended by design. Most accidents start at the operations level, where human controllers interact with a physical process, perhaps through an automated control system. Therefore, the analysis segment of CAST would start from there, examining the flaws in human and automated controllers, in conjunction with the context of the accident. Since the SCS at the operations level is always determined by managerial and regulatory decisions, the analysis would then proceed in a top-down manner, evaluating the positive or negative contributions of those decisions, as well as environmetnal factors such as economic pressures, political climates, and organizational safety culture, to the accident. Finally, CAST concludes its analysis segment by identifying the dynamics of the system from a safety perspective. Specifically, it identifies the manners in which the system's components exacerbated or diminished the risk factors, through establishing a system dynamics model and exposing the it to simulated scenarios. From there, CAST can help format the findings into an accident report, which can then be used to inform future safety measures and regulations.

\section{Safety Control Structure}
% Control Hierarchy: Map out the safety control structure, detailing how safety constraints were supposed to be enforced from design through operation.

% Communication Channels: Examine the communication pathways among stakeholders and how information about structural issues was shared or neglected.

% Regulatory Oversight: Assess the role of municipal and provincial regulations in ensuring building safety and how these may have failed.

\underline{System Goals}

G1. Provide a safe, functional, and accessible commercial environment for the residents and visitors of Elliot Lake

G2. Ensure the structural integrity of the building throughout its lifecycle

\underline{Losses}

L1. Human losses: loss of life, physical injuries and psychological harm

L2. Material and structural losses: destruction of property and economic losses

L3. Institutional and professional failures: loss of public trust, damage to the image of the engineering profession, and less trust in governmental accountability

L4. Disruption to society: loss of a commercial space, and breakdown of emergency responses

\underline{Hazards}

H1. Corrosion of the structure due to leaking from the rooftop parking deck [L1, L2, L3, L4]

H2. Ineffective inspection and maintenance routine [L1, L2, L3, L4]

H3. Inadequate engineering assessments [L1, L2, L3, L4]

H4. Regulatory and oversight gaps [L1, L2, L3, L4]

H5. Unclear responsibility allocation and poor communication [L1, L2, L3, L4]

H6. Economic and political pressures [L4]

H7, Unpreparedness when an emergency emerges [L1, L3, L4]

\underline{System Safety Requirements and Constraints}

SR1: The building must be designed and maintained to ensure long-term structural integrity under expected environmental and usage conditions. [H1]

SC1: Structural components must not be allowed to degrade to the point of losing load-bearing capacity. [H1]

SR2: Regular, thorough inspections of critical infrastructure (e.g., roof, support beams) must be conducted by qualified professionals. [H2]

SC2: Identified structural deficiencies must be documented transparently and acted upon within a reasonable time frame. [H2]

SR3: Engineering assessments must be objective, evidence-based, and prioritize public safety over client interests. [H3]

SC3: Engineers must communicate risk levels clearly and explicitly, including in cases where conditions are unsafe or immediate action is needed. [H3]

SR4: Municipal and provincial authorities must have mechanisms to enforce that the building safety standards are followed. [H4]

SC4: Regulatory bodies must step in when reports indicate serious structural risks, including through orders, fines, or building closure. [H4]

SR5: Property owners must allocate resources for ongoing maintenance and be held accountable for not doing repairs that pose safety risks. [H5]

SC5: Cost-saving measures must not go above the minimum safety standards required for public occupancy. [H5]

SR6: Emergency response plans must account for building collapse scenarios and provide for safe, timely rescue efforts. [H6]

SC6: Emergency response should not be suspended without an effective alternative strategy when lives may be at risk. [H6]

SR7: The public and tenants must be informed about any significant structural safety risks affecting the buildings they use. [H7]

SC7: Safety-related information must not be kept away from the public due to commercial, political, or reputational concerns. [H7]




\section{Event Analysis (tentative name)} % include causal loop and misc diagrams. RAUL
%accimap or (walkerton model -> TERESA)
% Timeline of Events: Construct a detailed timeline from the mall's construction to the collapse, highlighting key events such as inspections, reports of leaks, and maintenance actions.


% \end{itemize}
% \end{itemize}

% Failure Points: Identify specific points where safety constraints were breached, such as ignored engineering reports or inadequate repairs. RAUL

FAILURE POINTS

Ignored Persistent Warnings = Leaks and other issues began shortly after the mall opened. This was just one of the early indicators of the roof's poor structure. Not only did they ignore the vast number of leaks, but they also ignored a great warning in June of 2011 when a piece of concrete fell from the roof at the restaurant Hungry Jack. Even though there was an initial report filed, an inspector never arrived to do the inspection, allowing the negligent management to continue like no wrong was done.
 
Inadequate and Superficial roof repairs = Even with decades of complaints from various stakeholders, including customers and shops themselves, the mall management decided to use temporary and weak solutions to fix these issues instead of addressing the actual cause of the issues. As the saying goes, they were curing the symptoms but not the disease itself. As can be seen in the photo below, they were using nets to stop the leaks instead of addressing the root problem itself. This was due to the higher cost of more permanent and effective solutions.
  

 

 
Ignored / Rescinded city orders = Eliot Lake issued multiple warnings regarding the safety of customers and tenants. These warnings not only included the safety of these stakeholders but also included the maintenance of various issues. There was a notice of violation under the fire code and an order to remedy property standards. There were measures put in place due to these notices, but they were either inadequately enforced or built in order to be temporary solutions. This bad application of city order allowed the structural issues to persist and gave a false sense of safety for both management and the public.

 

 

 
Inadequate structural interventions = When the structural issues were being addressed, the repairs were either badly planned, insufficient, or incomplete. An example of this was how fireproofing was applied directly over the corroded beams without fixing those. In 2008/2009, there were plans to install a proper waterproofing membrane, but it was canceled due to cost concerns. It could be said that these fixes were mostly cosmetic rather than structural since they failed to address the growing corrosion problem.

 

 
Superficial Engineering Report by Robert Wood = In April of 2012 (two months before the roof collapse), engineer Robert Wood signed a report stating that there were no visual structural issues visible at the mall. It was later found that he never removed any finishes or investigated areas that were prone to corrosion. He was pressured by the mall’s owner to downplay his concerns which directly contributed to the false sense of safety, delaying any action that could have prevented the collapse.

 

 
Unchecked and widespread corrosion = Over 3o years, water leaks from the rooftop parking area caused steel beams and welds to rust severely. Investigations after the collapse indicated that some welds had corroded to only 12\% of their original strength and that approximately 40\% of structural connections showed deterioration. This corrosion remained unchecked, unreported, and ignored by the inspector who failed to identify the great danger this posed.

 

 
Roof Design flaws = Due to the idea that the mall had to be modern and visually pleasing, the original mall design included a rooftop parking above the retail stores, which exposed the mall's internal structure to water and salt intrusion. A waterproofing membrane that was recommended by the architect was omitted entirely in the construction. Over the years, the roof was subject to unorganized vehicular loads, which combined with snow accumulation and salt exposure, increased the wear and corrosion rate.

 

 
Faulty Material installation = Waterproofing materials and concrete slabs were often installed under poor conditions, such as cold or wet weather, which compromises their durability and effectiveness. From the beginning, there were construction shortcuts taken that contributed to the degradation of the structure. Another example of this is how they used hollow-core slabs in a high-exposure environment, such as the rooftop parking. Without proper and durable waterproofing, the integrity of these slabs was reduced significantly. The lack of reinforcement of these slabs made the collapse only a matter of time.

 

 

 
Uncontrolled vehicle loads = The rooftop parking was a source of structural overload. Vehicle traffic on the roof created weight loads that the roof was not constructed to support. Reports are indicating that heavy and unregulated traffic (snowplows and delivery trucks) used the roof without structural reinforcements in place.

 

 
Fabricated Maintenance records = Mall management and ownership (especially Bob Nazarian) submitted falsified maintenance invoices to the bank and lenders. These false documents included fake documentation for repairs that were never completed. This deception misled financial institutions but also prevented authorities from identifying and responding to risks from the mall's real conditions.

% Systemic Factors: Analyze how systemic issues, like organizational culture or economic pressures, contributed to the failure. RAUL

SYSTEMIC FACTORS

· Cost-cutting over safety = Throughout the mall's history, the theme of prioritizing saving money/profits over ensuring safety was a recurring occurrence. The most notable example is how its owner, Bob Nazarian, cancelled a \$1 million contract to install the proper waterproofing system since it seemed too expensive. To save money, he hired a contractor who gave them a lower price and even failed to pay them in full after the repairs were completed (which may also speak to the quality of said repairs). These decisions accelerated deterioration and directly compromised the structural integrity of the parking lot.

 

 · Weak regulatory oversight =The city of Elliot Lake failed to consistently enforce safety standards. Even after issuing multiple violations and hearing complaints from various sources, the city rescinded enforcement actions that they had placed previously, allowing Nazarian to neglect necessary repairs. During this time, inspectors failed to ensure code compliance. The leniency allowed unsafe conditions that remained unchecked and escalated through time.

 

 · Organizational negligence and poor maintenance culture = There was a long history of problems that were either ignored, patched superficially, or never even attempted to be repaired. It is clear how the mall was under a management team that treated repairs as something reactive instead of proactive. An example of this is how leaks were addressed with caulking instead of fixing the root cause of the leak. The complaints from various tenants were dismissed, which created a culture of negligence when it came to repairs and maintenance.

 

 · Weak communication among stakeholders = There was a systemic failure when communicating between vital stakeholders, including previous and current mall owners, engineers, city officials, and tenants. It is clear how the previous owners of the mall knew exactly the deficiencies of the structure and wanted to get rid of the liability. They assured the buyers the leaks were a small, fixable issue instead of the reality of the situation. Tenants reported major issues (like the piece of concrete that fell, and even though there was a complaint, the authorities never showed up,) and there was no follow-up from the mall. The broken communications between stakeholders led to no action being taken from authorities or other governing bodies who could have acted.

 

 · Ethical lapses among professionals = Several professional engineers involved in the mall ignored their duty to public safety. Robert Wood admitted he did not conduct a proper inspection and not only downplayed his concerns but also included false reassurances in his report. This lapse can be seen in multiple levels of the organization, starting with Nazarian, who had no moral ground and decided that the safety of the community was secondary to his profits. The use of hollow-core slabs is proof that professionals who know the limitations of the materials used, especially in humid/cold environments, show how much the lapse of judgement occurred. Whether this was for cost-saving measures, poor maintenance culture, or an inspector, everyone who knew the limitations of the materials and that they were not optimal for the environment where being used had failed in their professional duty either on purpose or by accident.

 

 · Surface-level inspections = Many of the inspections that were conducted of the structure and condition of the mall were done to a superficial level. The inspectors did not remove ceiling panels, and most importantly, they never tested for corrosion of the internal structure. All this allowed for deteriorated welds and extensive rust to go unnoticed and undocumented for decades. The mall was repeatedly declared structurally in good condition even with obvious external and internal issues to the structure.

 

 

 


 · Profit-driven concealment = Nazarian and previous mall owners hid the truth about the mall's conditions to maintain financial viability. As previously mentioned, there is evidence of falsified invoices to lenders to make the appearance that repairs had been implemented. Negative findings during inspections were either omitted from reports or softened so that the mall could continue to operate in its current conditions without having to make repairs and maintenance. By hiding the true structural deficiency of the mall and its risks, management protected the short-term financial viability of the mall while exposing the community to increasing danger.

 

 · Inadequate regulatory standards = The Ontario building code and inspection framework did not require invasive structural changes or account for specific risks (such as the rooftop parking). At the time, there was no regulatory guidance for long-term corrosion or structural integrity in environments with repeated water infiltration. Without the mandatory corrosion inspection or structural health monitoring, regulators had limited tools to enforce deeper reviews even with obvious and apparent structural issues. The 2012 Ontario Building Code was effective from January 1st, 2014. This code included provisions addressing corrosion, which is expected to be upheld in subsequent versions. 


\section{Causal Factors and Contributing Conditions}
% Immediate Causes: Detail the direct technical reasons for the collapse, such as corrosion of structural components. KEVIN

\subsection{Immediate Causes}

1. The Failed Beam-to-Column Connection at Gridline G-16 \\
By 2012, the beam-to-column connection at grid location G-16 had been degraded past the point of safety. This connection was part of the framing supporting the rooftop parking deck near the mall’s food court and escalator. The connection detail was a standard double-angle shear connection: “two steel angles bolted to the web of the W24×110 beam... and welded to the flange of the column at gridline intersection G-16. The failure occurred in the welded connection between the steel angles and the flange of the column.” In simple words, two L-shaped steel angle brackets were attached to the side of the column’s flange by welding, and the beam’s vertical web was then bolted to these angles. This kind of connection is pretty common and it allows the beam to carry vertical loads through the shear resistance of the angles and bolts. Under normal conditions, this connection would have a generous safety margin to support the parking deck loads.
However, the condition of the grid G-16 connection in 2012 was extremely poor. Decades of water intrusion had rusted it to an extreme degree. The Commission’s forensic report indicates that by the time of the collapse, “more than 85\% of the original weld capacity of the failed connection had been lost to corrosion.” This means that only approximately 15\% of its expected strength. The angles and bolts were also described as “severely corroded” in the later examination. Photographic evidence from Exhibit 3015 showed that the angle and bolted connection were already severely corroded. Essentially, the welds that connected the angle iron to the columns have become very fragile, covered with damaged parts and oxides.

Suprisingly, this connection and technical design did not have any potential risks or non-compliant sections. The Commission claims that: “The connection at gridline G-16 did not fail because of a construction defect. It failed because of exposure over the years to constant wetting and drying conditions in the presence of chlorides...” This means, the connection was originally built as intended, but it was not protected from the environment that it was subjected to. Neither the design nor the construction anticipated that this steel connection would be covered in salt water for years. Thus, while the design of the Algo Mall’s structure met the building code, an implicit design flaw was the lack of durability in wet conditions, especially in water that contians chloride. The inquiry report pointed to “those design inadequacies (including, in particular, the lack of a waterproof membrane as part of the original design or as a later addition)” which “led inevitably to the exceptional corrosion that precipitated the collapse.” In sum, the beam-to-column connection failed because it was heavily rusted, rather than  any structural, assembly, or construction issues. Once approximately 85\% of the weld’s cross-section was rusted, the remaining metal could no longer carry the normal loads of the parking deck structure.

2. Collapse Sequence and Structural Failure Mechanism \\
The collapse was triggered by the failure of a weak weld at the G-16 connection. The Commission’s engineering analysis and an animated video re-creation showed how the failure progressed: “On June 23, 2012, failure occurred in this steel connection which is located just below the deck… Without support, the concrete panels [of the parking deck] collapsed into the upper mall adjacent to the food court.” The end of the steel beam fell downward when its welded anchor broke, and the hollow concrete slab supported by the beam at that edge also collapsed. This concrete slab (and possibly several adjacent concrete slabs connected by the concrete above) fell onto the open space in the atrium of the shopping mall's second-floor. Shoppers on the second floor were hit by this piece of concrete and steel, and the impact was devastating: “Large pieces of steel and concrete (‘widow makers,’ the rescuers called them) hung precariously over the huge pile of debris” in the aftermath. 

The collapse did not stop after the initial collapse. According to the inquire, the falling debris caused a secondary structural failure in the mall’s second-floor. When the heavy rooftop slab and beam fell onto the second floor (the “upper mall”), they lead to another critical connection or beam on that level to fail, which in turn brought down a section of the second floor onto the ground floor. The Commission report refers to this as “a secondary collapse of a portion of the upper-level Mall framing”. As a result, the roof of the shopping mall and a roughly triangular or square area on the second floor collapsed all the way to the ground, leaving a large hole open to the sky.

3. Two-Stage Weld Failure and Material Analysis \\
In the course of the Commission’s inquiry, structural engineers, metallurgists, and corrosion specialists performed a detailed analysis on the failed connection. Their findings revealed that the weld failure did not happen instantaneous, but rather a “two-stage” failure that had begun some time before the collapse. Specifically, evidence of an earlier, partial crack was found on the weld surface. NORR reported that the weld had likely broken months before June 2012, but had remained partially intact until the day of collapse.
The report notes that the presence of pitting and black oxide on some portions of the broken weld, and the absence of such corrosion on other portions. This pitting and black rust indicate that a crack surface had been exposed to the environment for a really long period. In the connection part, one portion of the weld fracture showed these features, this implys that this part of the weld was damaged earlier, and then kept corroding for a while. In contrast, another portion of the final breaking surface was clean, which implys a sudden break at the moment of collapse. NORR’s experts then concluded that the weld had experienced an initial failure before the collapse, the remaining section then suffered further corrosion across months. Eventually, when the residual intact part  could not support the load, the final collapse occured.
Additionally. A small triangular piece of the steel was found, and was still attached to the column flange after the collapse. Upon microscopic inspection, engineers observed that this piece have been pried or bent off, rather than a sudden break. This analysis strongly confirms that the collapse was not caused by sudden overload, but was the result of a long-term process.


% Underlying Systemic Causes: Explore deeper systemic issues, including decision-making processes, accountability structures, and resource allocations that allowed the immediate causes to develop. KEVIN


\subsection{Underlying Systemic Causes}

1. Structural Design of the Algo Centre Mall and Its Roof System \\
The Algo Centre Mall was a steel-framed building with a parking deck on the roof, and it was opened in 1979. The building was made of hollow-core concrete slab panels for both roof and floors, and the steel columns and beams were used to support them. The rooftop parking deck was built with 8-inch-thick hollow-core concrete slabs topped with a cast-in-place concrete wearing surface (topping) for additional strength and to protect the panels. This topping was used to help the roof deck achieve the required load capacity for parking (approximately 120 pounds per square foot). The Commission found that the overall structural design technically met the 1975 Ontario Building Code requirements. However, meeting the bare minimum requirements was not sufficient in this case.
A notable feature of the mall’s original design was its unconventional waterproofing strategy for the rooftop parking deck. Instead of incorporating a continuous membrane or a sloped drainage layer, the roof was designed as a flat concrete deck with a surface-applied waterproofing sealer to keep water out. The architectural drawings only indicated a “waterproofing sealer” on top of the concrete, but the specific product was not clearly specified. In principle, the 1975 Building Code required that a roof should “shed or drain water effectively", but the mall’s designers simply addressed this by calling for a waterproof coating. In practice, the roof failed to remain watertight from the very beginning. As one of the NORR members concluded, the design “narrowly meets the requirements of Part 4 of the OBC (1975) but relies entirely upon the ‘WATERPROOFING SEALER’ material”, this means that this design could not survive for a long time in a severe climate.
In addition to the design,  the waterproofing system's implementation during construction was extremely poor. The evidence shows that “the waterproofing of the roof failed virtually from the outset”. From the very first day the mall opened, water started to penetrate the rooftop deck. The mall’s roof was flat and supposed to drain water as quickly as possible, but in fact, it did not appear to have happened. The joints between the slabs, the small cracks in the concrete top and the interfaces around the drains or the expansion joints became pathways for the ingress of water. Within weeks of opening, there was already evidence of leaks throughout the building. In summary, the mall’s roof system was fundamentally flawed. The building lacked any membrane design or drainage strategy to manage this condition caused by water and salt. This design deficiency set the stage for the corrosive destruction of the mall’s structural components over its years.

2. Chronic Water Leakage and Corrosion of Structural Components \\
The immediate technical cause of the Algo Mall collapse was the severe corrosion at a critical steel connection, and the corrosion on these components is the direct result of the uncontrolled water leakage. The Commission’s investigation states that leaks at the exact location of the ultimate collapse (gridline 16) were observed as early as August 1981. Over the next decades, water infiltrated along the gridline G-16 area, which corresponded to a column line supporting a portion of the rooftop parking slab. As mentioned before, during winters, cars brought in snow and road salt, and every spring, salt-water would seep through the cracked topping concrete. The inquiry report states that the connection on the G-16 gridline “failed due to exposure over the years to constant wetting and drying conditions in the presence of chlorides, which leaked from the rooftop parking deck unabated for more than 30 years”. Therefore, this slow but continuous corrosion led to the eventual collapse.
The penetrating water carried dissolved road salt that contains chlorid from the parking deck, creating a brine that reached the steel beams, columns, and connectors. These steel components were not designed for use in a wet, salt-contaminated environment. The chlorides from road salt used to melt snow would significantly damage concrete structures and the embedded steel strands within hollow slabs. Chlorides can also penetrate the concrete, reducing its alkaline properties, destroying the protective layer around the steel strands and causing a loss of section. Once this protective layer is damaged, the steel strands start to corrode, and lose their original strength and integrity. Since Corrosion increases the volume of the steel strands, exerting internal pressure, it causes the concrete to crack and break. The mall occupants claim that these damages were even clearly visible, and they took photographs of these parts.

% Human and Organizational Factors: Assess how human behavior, organizational culture, and management decisions contributed to the incident. RAUL

\section{Recommendations}
% Design and Maintenance: Suggest improvements in building design practices and maintenance protocols to prevent similar failures. TERESA

% Regulatory Reforms: Propose changes to regulatory frameworks to enhance oversight and enforcement of safety standards. KEVIN

\subsection{Regulatory Reforms:}

\subsubsection{Strengthening Building Inspections and Maintenance}

One of the most noticeable issues with Ontario's current system, is the absence of rigours and continuous inspections for existing structures. Under the status quo, once a building has passed its initial occupancy inspection, mandatory periodic structural reviews are not typically conducted, and issues are usually only addressed when complaints or obvious hazards appear \cite{DoodyRemarksPEO}. In Algo Centre Mall's case, over 33 years, multiple warning signs (leaks, rust, even falling concrete) were either ignored or addressed with superficial fixes; many inspections were very simple and meaningless. As the Commission stated, “inadequate or no inspection for structural safety” was a major factor leading to the collapse. Inspectors did not check the structure inside the ceiling panels or corrosion test, thus missing the severe rusting that weakened the steel frame. In some cases, negative findings were even hidden, omitted or softened in reports to avoid costly repairs. These issues show that Ontario urgently needs to change how it checks and supervises building inspections and maintenance. To address this issue, the Ontario provincial government should institute a comprehensive Periodic Building Safety Inspection program for existing buildings:

1. Mandatory Structural Inspections at Regular Intervals: Structural inspections must be strictly enforced and conducted regularly. All large or aging buildings that are open to the public should undergo thorough structural inspections by qualified engineers on a fixed cycle (e.g. every 5 years). The Elliot Lake Commission explicitly considered “mandatory periodic inspection of all buildings in the province” by either the owner or public authorities. Regular building status updates would ensure problems like water penetration and corrosion are caught early before they escalate. Commissioner Paul Bélanger recommended that building owners should be required to ensure not only that their structures are initially safe but that they remain safe throughout their useful life. This obligation can be enforced by requiring owners to submit structural reports summarized by authorized agencies regularly to prove the stability of the building. These reports would point out any signs of deterioration and certify that the building meets at least the minimum structural standards at the time of inspection.

2. Empowering Building Officials: Even the best standards mean little without enforcement. Therefore, Ontario should strengthen the powers of local building officials, while preventing abuse of power and corruption, so that they can intervene when safety hazards are identified during inspections. The inquiry proposed “increased powers for all building officials… to make orders with respect to buildings that are or could become unsafe”. In practice, this could allow Chief Building Officials (CBO) to require preventive repairs or further expert assessments if an inspection report indicates potential structural issues. Building officials might also be empowered to mandate an independent peer review of a suspect building’s condition. Additionally, it is essential to improve the training and qualifications of building inspectors and property standards officers. The Commission noted that some officials in Elliot Lake took a narrowly bureaucratic view and missed the bigger safety picture. Enhanced training in building safety  would enable inspectors to recognize potential hazards and enforce standards more effectively. 



\subsubsection{Enhancing Professional Accountability and Standards}

Another key area for reform is the accountability of professionals (engineers, architects, etc.). Engineers have a duty to put public safety first in building design and assessment. However, the Elliot Lake investigation found that some engineers forgot (or ignored) this moral and ethical foundation. In the decades before the collapse, multiple engineering reports on the Algo Mall either downplayed the severity of the structural problems or failed to press for the necessary fixes. If engineers can sign off on unsafe structures with impunity or without rigorous standards, the public is left at risk. Thus, strengthening professional accountability is vital:


1. Ethical Duties: Engineers and architects should be obligated to report any serious safety hazard they encounter, even if it means displeasing a client or employer. Current engineering ethics standards do indeed prioritize public safety, but the Elliot Lake case shows that in practice, engineers feel pressure (or are willing) to downplay warnings. In the future, regulatory agencies should issue clear instructions specifying that if engineers discover imminent dangers (such as the risk of collapse), they must clearly inform the owner in writing and notify the relevant departments. The Commission in fact recommended that engineering reports’ contents should not be altered due to client pressure, and that PEO should explicitly inform its members such requirements. In addition, The Ontario provincial government could introduce whistleblower protections for certain professionals: if engineers report safety issues in good faith to the authorities (e.g., bypassing unresponsive owners), they should be protected from legal or professional repercussions. This encourages professionals to act in the public interest first and foremost.

2. Qualified Structural Engineers and Standardized Reporting: These critical inspections should be performed by professionals and experts from certified institutions or third parties. The inquiry found that some engineers who assessed the Algo Mall lacked the required diligence or expertise. Some reports of the Algo Centre Mall were so cursory as to be essentially meaningless. To prevent substandard assessments, such structural inspections must be conducted by appropriately qualified structural engineers, especially those with specific qualifications or certifications in structural safety. The Commission urged the Association of Professional Engineers of Ontario (PEO) to “toughen its standards” for licensing engineers who conduct structural inspections. One concrete measure would be creating a “structural engineering specialist” designation, as recommended in the inquiry report, with defined qualifications and a code of practice for evaluating existing buildings. In addition to certification, PEOs should also establish clear performance standards that specify the appropriate scope of structural inspections in detail. For example, engineers should be required to inspect critical connections for corrosion and detect known leak areas, rather than relying simply on visual observation. This was actually a commission recommendation: “PEO should enunciate a Performance Standard for the prescribed structural inspection” to guide practitioners. Every inspection report should also follow a consistent template, including all the investigative steps that are performed, all the evident and factual findings and the final investigation result, to reduce the chance of oversight or intentional downplaying of issues.



% Organizational Changes: Recommend organizational reforms to improve communication, accountability, and safety culture among stakeholders. RAUL

\section{Aftermath}
%TERESA

\section{Conclusion}
% Summary of Findings: Recap the key insights gained from the CAST analysis.

% Implications for Future Practice: Discuss how these findings can inform future practices in building design, maintenance, and regulation.

\section*{Appendices}

\newpage
\printbibliography
\end{document}
