\documentclass[12pt]{article}
\linespread{1.2}
\setlength{\parskip}{1em}
\usepackage{hyperref}
\usepackage{xpatch}
\usepackage[sorting=none]{biblatex}
\usepackage{fancyhdr}
\usepackage[a4paper, margin=1.25in]{geometry}
\addbibresource{references.bib}
\hypersetup{
    colorlinks=true,
    linkcolor=black,
    filecolor=magenta,      
    urlcolor=black,
    citecolor=black
    }
\urlstyle{same}

\title{}
\date{June 2025}

% Here you can find a backup of this project: 
% https://github.com/ThisisShoo/APS-1034-Team-Project-Report

% Lucid Chart: 
% https://lucid.app/lucidchart/f0b534e7-64c4-4566-ac8d-bf4807728f79/edit?invitationId=inv_bbe79a52-9dfe-407f-94ed-31a270cd020d&page=0_0#

\begin{document}
\maketitle


\newpage

\section*{Executive Summary}
% Purpose: Summarize the objectives, key findings, and recommendations of the analysis.

% Overview: Provide a brief description of the Algo Centre Mall collapse, including the date, location, and immediate consequences.

\section{Introduction} 
% - History of the mall, chain of events, as well as the methodology for analysis 
% Background: Detail the history of the Algo Centre Mall, including its construction in 1979-1980, architectural features like the rooftop parking, and its significance to the Elliot Lake community.
% Incident Summary: Outline the events leading up to the collapse on June 23, 2012, the collapse itself, and the immediate aftermath.
% Scope and Methodology: Explain the CAST methodology and its relevance to analyzing this particular incident.

The Algo Centre Mall, located in Elliot Lake, Ontario, was constructed between 1979 and 1980 by Algocen Realty Holdings Ltd. Housing a public library, a government service center, a hotel and retirement residence, and various retail stores, the mall served not only as a commercial hub but also as a community gathering space. The three-story structure was composed of a steel frame supporting concrete and steel-reinforced concrete slabs, sitting on a sloped terrain facing northeast. A parking deck was built on the roof, which was accessible via ramps along the hillside. The parking deck was supported by a steel frame held together by bolted and welded connections, with a layer of hollow core concrete slabs on top. 

Throughout the years almost immediately following its construction, the mall experienced significant structural issues, particularly with water leaks from the rooftop parking deck. Such water leaks, along with cracks in the concrete slabs, were reported as early as 1981, and the building's condition continued to deteriorate over the years. In 1991, a report prepared by Trow Consulting Engineers Ltd. found that although the rooftop parking deck was generally in good condition, various components of the building had visible signs of deterioration, including the aftermath of failed repairs, sections of broken concrete, and surface rust on exposed steel. 


\cite{AlgoLakeReport1}

\section{Safety Control Structure}
% Control Hierarchy: Map out the safety control structure, detailing how safety constraints were supposed to be enforced from design through operation.

% Communication Channels: Examine the communication pathways among stakeholders and how information about structural issues was shared or neglected.

% Regulatory Oversight: Assess the role of municipal and provincial regulations in ensuring building safety and how these may have failed.

\underline{System Goals}

G1. Provide a safe, functional, and accessible commercial environment for the residents and visitors of Elliot Lake

G2. Ensure the structural integrity of the building throughout its lifecycle

\underline{Losses}

L1. Human losses: loss of life, physical injuries and psychological harm

L2. Material and structural losses: destruction of property and economic losses

L3. Institutional and professional failures: loss of public trust, damage to the image of the engineering profession, and less trust in governmental accountability

L4. Disruption to society: loss of a commercial space, and breakdown of emergency responses

\underline{Hazards}

H1. Corrosion of the structure due to leaking from the rooftop parking deck [L1, L2, L3, L4]

H2. Ineffective inspection and maintenance routine [L1, L2, L3, L4]

H3. Inadequate engineering assessments [L1, L2, L3, L4]

H4. Regulatory and oversight gaps [L1, L2, L3, L4]

H5. Unclear responsibility allocation and poor communication [L1, L2, L3, L4]

H6. Economic and political pressures [L4]

H7, Unpreparedness when an emergency emerges [L1, L3, L4]

\underline{System Safety Requirements and Constraints}

SR1: The building must be designed and maintained to ensure long-term structural integrity under expected environmental and usage conditions. [H1]

SC1: Structural components must not be allowed to degrade to the point of losing load-bearing capacity. [H1]

SR2: Regular, thorough inspections of critical infrastructure (e.g., roof, support beams) must be conducted by qualified professionals. [H2]

SC2: Identified structural deficiencies must be documented transparently and acted upon within a reasonable time frame. [H2]

SR3: Engineering assessments must be objective, evidence-based, and prioritize public safety over client interests. [H3]

SC3: Engineers must communicate risk levels clearly and explicitly, including in cases where conditions are unsafe or immediate action is needed. [H3]

SR4: Municipal and provincial authorities must have mechanisms to enforce that the building safety standards are followed. [H4]

SC4: Regulatory bodies must step in when reports indicate serious structural risks, including through orders, fines, or building closure. [H4]

SR5: Property owners must allocate resources for ongoing maintenance and be held accountable for not doing repairs that pose safety risks. [H5]

SC5: Cost-saving measures must not go above the minimum safety standards required for public occupancy. [H5]

SR6: Emergency response plans must account for building collapse scenarios and provide for safe, timely rescue efforts. [H6]

SC6: Emergency response should not be suspended without an effective alternative strategy when lives may be at risk. [H6]

SR7: The public and tenants must be informed about any significant structural safety risks affecting the buildings they use. [H7]

SC7: Safety-related information must not be kept away from the public due to commercial, political, or reputational concerns. [H7]


\section{Event Analysis (tentative name)} % include causal loop and misc diagrams. RAUL / TERESA

% Timeline of Events: Construct a detailed timeline from the mall's construction to the collapse, highlighting key events such as inspections, reports of leaks, and maintenance actions.

% Failure Points: Identify specific points where safety constraints were breached, such as ignored engineering reports or inadequate repairs.

% Systemic Factors: Analyze how systemic issues, like organizational culture or economic pressures, contributed to the failure.

\section{Causal Factors and Contributing Conditions}
% Immediate Causes: Detail the direct technical reasons for the collapse, such as corrosion of structural components. RAUL / TERESA


% Underlying Systemic Causes: Explore deeper systemic issues, including decision-making processes, accountability structures, and resource allocations that allowed the immediate causes to develop.

% Human and Organizational Factors: Assess how human behavior, organizational culture, and management decisions contributed to the incident.

\section{Recommendations}
% Design and Maintenance: Suggest improvements in building design practices and maintenance protocols to prevent similar failures. RAUL / TERESA

% Regulatory Reforms: Propose changes to regulatory frameworks to enhance oversight and enforcement of safety standards.

% Organizational Changes: Recommend organizational reforms to improve communication, accountability, and safety culture among stakeholders.

\section{Aftermath}
%RAUL / TERESA

\section{Conclusion}
% Summary of Findings: Recap the key insights gained from the CAST analysis.

% Implications for Future Practice: Discuss how these findings can inform future practices in building design, maintenance, and regulation.

\section*{Appendices}

\newpage
\printbibliography
\end{document}
